%%%%%%%%%%%%%%%%%%%%%%%%%%%%%%%%%%%%%%%%%
% Beamer Presentation
% LaTeX Template
% Version 1.0 (10/11/12)
%
% This template has been downloaded from:
% http://www.LaTeXTemplates.com
%
% License:
% CC BY-NC-SA 3.0 (http://creativecommons.org/licenses/by-nc-sa/3.0/)
%
%%%%%%%%%%%%%%%%%%%%%%%%%%%%%%%%%%%%%%%%%

%----------------------------------------------------------------------------------------
%	PACKAGES AND THEMES
%----------------------------------------------------------------------------------------

\documentclass{beamer}

\mode<presentation> {

% The Beamer class comes with a number of default slide themes
% which change the colors and layouts of slides. Below this is a list
% of all the themes, uncomment each in turn to see what they look like.

%\usetheme{default}
%\usetheme{AnnArbor}
%\usetheme{Antibes}
%\usetheme{Bergen}
%\usetheme{Berkeley}
%\usetheme{Berlin}
%\usetheme{Boadilla}
%\usetheme{CambridgeUS}
%\usetheme{Copenhagen}
%\usetheme{Darmstadt}
%\usetheme{Dresden}
%\usetheme{Frankfurt}
%\usetheme{Goettingen}
%\usetheme{Hannover}
%\usetheme{Ilmenau}
%\usetheme{JuanLesPins}
%\usetheme{Luebeck}
\usetheme{Madrid}
%\usetheme{Malmoe}
%\usetheme{Marburg}
%\usetheme{Montpellier}
%\usetheme{PaloAlto}
%\usetheme{Pittsburgh}
%\usetheme{Rochester}
%\usetheme{Singapore}
%\usetheme{Szeged}
%\usetheme{Warsaw}

% As well as themes, the Beamer class has a number of color themes
% for any slide theme. Uncomment each of these in turn to see how it
% changes the colors of your current slide theme.

%\usecolortheme{albatross}
%\usecolortheme{beaver}
%\usecolortheme{beetle}
%\usecolortheme{crane}
%\usecolortheme{dolphin}
%\usecolortheme{dove}
%\usecolortheme{fly}
%\usecolortheme{lily}
%\usecolortheme{orchid}
%\usecolortheme{rose}
%\usecolortheme{seagull}
%\usecolortheme{seahorse}
%\usecolortheme{whale}
%\usecolortheme{wolverine}

%\setbeamertemplate{footline} % To remove the footer line in all slides uncomment this line
%\setbeamertemplate{footline}[page number] % To replace the footer line in all slides with a simple slide count uncomment this line

%\setbeamertemplate{navigation symbols}{} % To remove the navigation symbols from the bottom of all slides uncomment this line
}

\usepackage{graphicx} % Allows including images
\usepackage{booktabs} % Allows the use of \toprule, \midrule and \bottomrule in tables

%----------------------------------------------------------------------------------------
%	TITLE PAGE, [1]
%----------------------------------------------------------------------------------------

\title[Feature Noising Log-LS Prediction]{Supplementary Matarials} % The short title appears at the bottom of every slide, the full title is only on the title page

\author{Mose Park} % Your name
\institute[UOS] % Your institution as it will appear on the bottom of every slide, may be shorthand to save space
{
Department of Statistics \\
\medskip
\textit{University of Seoul}
}
\date{\today} % Date, can be changed to a custom date

\begin{document}

\begin{frame}
\titlepage % Print the title page as the first slide
\end{frame}

\begin{frame}
\frametitle{Overview} % Table of contents slide, comment this block out to remove it
\tableofcontents % Throughout your presentation, if you choose to use \section{} and \subsection{} commands, these will automatically be printed on this slide as an overview of your presentation
\end{frame}

%----------------------------------------------------------------------------------------
%	PRESENTATION SLIDES, [2]
%----------------------------------------------------------------------------------------

\section{Dropout}
\subsection{base figure}

%------------------------------------------------
\section{Conditional Random Fields} % Sections can be created in order to organize your presentation into discrete blocks, all sections and subsections are automatically printed in the table of contents as an overview of the talk
%------------------------------------------------

\subsection{Intuitive Explanation} % A subsection can be created just before a set of slides with a common theme to further break down your presentation into chunks
\subsection{Definition}
%------------------------------------------------
\section{Dynamic Programming}
\subsection{Explaination}
\subsection{Example}


%----------------------------------------------------------------------------------------
%	PRESENTATION SLIDES, [3]
%----------------------------------------------------------------------------------------
\begin{frame}[fragile] % Need to use the fragile option when verbatim is used in the slide
\frametitle{Dropout}

\centerline{\includegraphics[width=12.5cm]{fig.2_dropout.png}}

\end{frame}

%----------------------------------------------------------------------------------------
%	PRESENTATION SLIDES, [4]
%----------------------------------------------------------------------------------------
\begin{frame}
\frametitle{Conditional Random Fields}
\centerline{\includegraphics[width=8.5cm]{fig.1_CRFs.png}}

\end{frame}

%----------------------------------------------------------------------------------------
%	PRESENTATION SLIDES, [5]
%----------------------------------------------------------------------------------------
\begin{frame}
\frametitle{Definition}
\begin{Definition}[Linear-chain CRFs]

Let $X, Y$ be random vectors, $\theta = \{\theta_k\} \in \mathbb{R}^K$ be a parameter vector, and $\{f_k(y, y_0, x_t)\}_{k=1}^{K} $ be a set of real-valued feature functions.\\
Then \textit{a linear-chain conditional random field} is a distribution $p(y|x)$ that takes the form:

\begin{align*}
    p(y \vert x) = \frac{1}{Z(x)} \prod_{t=1}^{T} \exp\left(\sum_{k=1}^{K} \theta_k f_k(y_t, y_{t-1}, x_t)\right)    
\end{align*}
   

where $Z(x)$ is an instance-specific normalization function

\begin{align*}
    Z(x) = \sum_{y \in Y} \prod_{t=1}^{T} \exp \left( \sum_{k=1}^{K} \theta_k f_k(y_t, y_{t-1}, x_t) \right)    
\end{align*}


\end{Definition}


\end{frame}



%------------------------------------------------
%----------------------------------------------------------------------------------------
%	PRESENTATION SLIDES, [6]
%----------------------------------------------------------------------------------------
\begin{frame}
\frametitle{Dynamic Programming}


\begin{block}{Dynamic Programming}
In mathematics, computer science, and economics, dynamic programming refers to a method of solving complex problems by breaking them down into simple subproblems.
\end{block}
\end{frame}

%----------------------------------------------------------------------------------------
%	PRESENTATION SLIDES, [7]
%----------------------------------------------------------------------------------------
\begin{frame}[fragile] % Need to use the fragile option when verbatim is used in the slide
\frametitle{Recursion Example}
\begin{example}[Fibonacci sequence, Python]
\begin{verbatim}
def fib(n):
    if n==1 or n==2:
        return 1
    else:
        return (fib(n-1)+fib(n-2))

summary : 129884KB, 1828ms
\end{verbatim}
\end{example}
\end{frame}


%----------------------------------------------------------------------------------------
%	PRESENTATION SLIDES, [8]
%----------------------------------------------------------------------------------------
\begin{frame}[fragile] % Need to use the fragile option when verbatim is used in the slide
\frametitle{DP Example}
\begin{example}[Fibonacci sequence, Python]
\begin{verbatim}
def fib(n):
    f = [0] * (n+1)
    f[1] = f[2] = 1
    for i in range(3,n+1):
        f[i] = f[i-1]+f[i-2]
    return f[n]

summary : 31256KB, 44ms
\end{verbatim}
\end{example}
\end{frame}

\end{document}
